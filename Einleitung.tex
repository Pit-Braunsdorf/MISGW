
  	\chapter{Einleitung}
	\label{ch:einleitung}
	\section{Gegenstand und Motivation}
	\label{sec:gegenstand_und_motivation}
	\subsection{Gegenstand}
	\label{subsec:gegenstand}
		In der Architektur und dem Management von Informationssystem im Gesundheitswesen werden bestimmte Begriffe regelm��ig verwendet. Teilweise auch mit anderen Definitionen als in einem Lexika. Um das Verstehen der Begrifflichkeit und deren Zusammenh�nge zu vereinfachen, wurde eine Ontologie erstellt, welche diese sowohl visuell als auch schematisch darstellt. Diese wird \ac{SNIK} genannt. 
		Der Umfang von \ac{SNIK} beruht auf verschiedenen B�chern, die f�r die Ontologie bearbeitet wurden. Eines dieser B�cher wird als Lehrbuch f�r die zwei Module Architektur bzw. Management von Informationssystemen im Gesundheitswesen, welche am \ac{IMISE} angeboten werden, verwendet. Daher wird \ac{SNIK} au�erdem herangezogen, um eine �bersicht zu bieten, in wie weit diese Begriffe zu einander in Beziehung stehen. Diese �bersicht ist bei \ac{SNIK} visuell dargestellt. Die Visualisierung ist \href{www.snik.eu/graph}{hier} einsehbar. 
		
	\subsection{Problematik}
	\label{subsec:problematik}
		Da die Ontologie von \ac{SNIK} sehr umfangreich ist, ist die Visualisierung von \ac{SNIK} un�bersichtlich. Durch diese Un�bersichtlichkeit stellt sich die Frage, ob diese Visualisierung von \ac{SNIK} ihr Ziel erf�llt. Das Ziel ist eine Unterst�tzung von Studenten und Dozenten in der Lehre am \ac{IMISE} und eine vereinfachte Kommunikation. Dies kann anhand der Grunds�tze, die im Paper \glqq Affective Graphs: The Visual Appeal of Linked Data\grqq{} von Garcia et al. dargestellt werden, �berpr�ft und analysiert werden.
	\subsection{Motivation}
	\label{subsec:motivation}
		Anhand dieser Grunds�tze kann notfalls eine Vereinfachung der Visualisierung f�r die Lehre am \ac{IMISE} und f�r andere Verwendungen wie beispielsweise Kommunikation mit dem Universit�tsklinikum Leipzig erfolgen. Aus diesem Grund kann die Ontologie verwendet werden, um bestimmte Definitionen zu lernen bzw. um mit anderen auf Basis von diesen Definitionen zu kommunizieren. Dadurch wird ein gemeinsamen Wortschatz f�r die Kommunikation erm�glicht. Au�erdem wird erreicht, dass verschiedene Anwendungen auf diese Ontologie zur�ckgreifen k�nnen. Dadurch unkompliziertere Kommunikation zwischen Anwendungen und zwischen der Anwendung und einem Anwender sicher gestellt wird.
		
	\section{Problemstellung}
	\label{subsec:problemstellung}
		Die Problematik welche im Kapitel \ref{subsec:problematik} dargestellt wird, soll im Zuge dieser Arbeit auf die Ontologie \ac{SNIK} angewandt werden. Dabei soll die Visualisierung von \ac{SNIK} (einsehbar unter: \href{www.snik.eu/graph}{www.snik.eu/graph}) anhand der Grunds�tze aus dem Paper analysiert und entsprechende Schwachstellen identifiziert und analysiert werden. Die Analyse der Schwachstellen soll eine Einsch�tzung des Schweregrades und des gesch�tzten Behebungsaufwandes beinhalten.