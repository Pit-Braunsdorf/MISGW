\documentclass[12pt,a4paper]{scrreprt}
\usepackage[latin1]{inputenc}
\usepackage{amsmath}
\usepackage{amsfonts}
\usepackage{amssymb}
\usepackage{graphicx}
\usepackage{acronym}
\usepackage{hyperref}
\author{Pit Braunsdorf}
\begin{document}
	\chapter{Einleitung}
	\section{Gegenstand und Motivation}
	\subsection{Gegenstand}
		In der Architektur und dem Management von Informationssystem im Gesundheitswesen werden bestimmte Begriffe regelm��ig verwendet, teilweise auch mit einer anderen Definition als �blich. Um das Verstehen der Begrifflichkeit und deren Zusammenh�nge zu vereinfachen, wurde eine Ontologie erstellt, welche diese sowohl visuell als auch schematisch darstellt. Diese Ontologie wird \ac{SNIK} genannt. 
		Der Umfang von \ac{SNIK} beruht auf verschiedenen B�chern, die jeweils eingelesen wurden. Eines dieser B�cher wird als Lehrbuch f�r die Module Architektur bzw. Management von Informationssystemen im Gesundheitswesen verwendet. Daher wird au�erdem \ac{SNIK} verwendet, um eine �bersicht zu bieten, in wie weit diese Begriffe zu einander in Beziehungen stehen.
		%TODO �blich ist doof
	\subsection{Problematik}
		Wie bereits erw�hnt ist \ac{SNIK} eine sehr umfangreiche Ontologie, so dass eine �bersichtliche Visualisierung nur schwierig machbar ist. Daher soll im Zuge dieser Arbeit eine Analyse der bereits vorhandenen Visualisierung von \ac{SNIK} (einsehbar unter: \href{ www.snik.eu/graph}{www.snik.eu/graph}). Die Analyse soll anhand des Papers ...
	\subsection{Motivation}
		Durch die Verwendung von Ontologien in der Lehre ist es notwendig, dass diese sinnvoll strukturiert und visuell dargestellt werden k�nnen. 
		
	
	\begin{acronym}
		\acro{SNIK}{Semantisches Netz des Informationsmanagments im Krankenhaus}
	\end{acronym}
\end{document}
