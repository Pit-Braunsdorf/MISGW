
\chapter{Einleitung}
\label{ch:einleitung}
\section{Gegenstand und Motivation}
\label{sec:gegenstand_und_motivation}
\subsection{Gegenstand}\todo{Der Gegenstand ist zu schwammig beschrieben. F�r die SNIK-Ontologie bitte mal die Publikationen unter \url{http://www.snik.eu/de/Publikationen/index.jsp} durchsehen und die Relevanten zitieren.}
\label{subsec:gegenstand}
In der Architektur und dem Management von Informationssystem im Gesundheitswesen werden bestimmte Begriffe regelm��ig verwendet. Teilweise auch mit anderen Definitionen als in einem Lexikum.
Um das Verstehen der Begrifflichkeit und deren Zusammenh�nge zu vereinfachen, sowie die M�glichkeit zu erschaffen, dass diese Daten maschinell verarbeitet und mittels \ac{SPARQL} abgefragt werden k�nnen, wurde eine Ontologie erstellt. Diese wird \ac{SNIK} genannt.

Der Umfang von \ac{SNIK} beruht auf verschiedenen B�chern, die f�r die Ontologie bearbeitet wurden.
Eines dieser B�cher wird als Lehrbuch f�r die zwei Module Architektur bzw. Management von Informationssystemen im Gesundheitswesen, welche am \ac{IMISE} angeboten werden, verwendet.
Daher wird \ac{SNIK} au�erdem herangezogen, um eine �bersicht zu bieten, in wie weit diese Begriffe zu einander in Beziehung stehen.
Diese �bersicht ist bei \ac{SNIK} visuell dargestellt.
Die Visualisierung ist \href{http://www.snik.eu/graph}{hier} einsehbar.
Diese dient zur Unterst�tzung von Studierenden und Dozenten, bei der Erlangung des fachspezifischen Wortschatzes.
Au�erdem kann die Visualisierung verwendet werden, um Daten mittels \ac{SPARQL} abgefragt werden.
Dies unterst�tzt wiederum die Studierenden beim Lernen des fachspezifischen Wortschatzes, da dadurch spezielle, noch unbekannte, Worte gefunden werden k�nnen und die Definition der Worte erlernt werden kann.(vgl. \cite{Schaaf.})

\subsection{Problematik}
\label{subsec:problematik}
Die Visualisierung von \ac{SNIK} beruht auf einer graphischen Darstellung der Begriffe und deren Beziehungen. 
Durch die Vielzahl von Begriffen und Beziehungen ist diese graphische Darstellung sehr un�bersichtlich.
Durch diese Un�bersichtlichkeit stellt sich die Frage, ob diese Visualisierung von \ac{SNIK} ihr Ziel erf�llt.
Das Ziel ist eine Unterst�tzung von Studenten und Dozenten in der Lehre am \ac{IMISE} und eine vereinfachte Kommunikation zwischen Anwendungen und Anwendern bzw. zwischen Informationstechnikern.
Dies kann anhand der Grunds�tze, die im Paper \glqq Affective Graphs: The Visual Appeal of Linked Data\grqq{} von Garcia et al. dargestellt werden, �berpr�ft und analysiert werden.

\subsection{Motivation}
\label{subsec:motivation}
Die Qualit�t von Linked Data ergibt sich aus der Verwendbarkeit f�r Menschen und Maschinen (vgl. \cite{Garcia.}). 
Durch die bereits erw�hnte Un�bersichtlichkeit ist es fragw�rdig, ob die Verwendbarkeit von der Visualisierung von \ac{SNIK} f�r Menschen leicht verst�ndlich und einfach ist. 
Dadurch wird die Qualit�t von \ac{SNIK} verringert. 
Damit die Verwendbarkeit von \ac{SNIK} f�r Menschen und Maschinen, damit auch dessen Qualit�t, verbessert werden kann, m�ssen Schwachstellen identifiziert werden.
Die Identifizierung von Schwachstellen kann anhand der Grunds�tze aus dem Paper \cite{gutefrage.de} durchgef�hrt werden. 
Sind die Schwachstellen erst identifiziert, k�nnen diese auch behoben werden. 

Als Ergebnis der Verbesserung der Qualit�t ist eine Verbesserung der Visualisierung von \ac{SNIK} zu erwarten, wo durch eine einfachere Kommunikation durch einen gemeinsamen Wortschatz m�glich wird.
Dieser gemeinsame Wortschatz kann durch die Verbesserung der Visualisierung leichter erlernt werden.
Dadurch dass auch Maschinen auf diese Ontologie zugreifen k�nnen, kann durch eine Verbesserung der Qualit�t auch eine unkompliziertere Kommunikation zwischen Anwendungen und Anwendern sicher gestellt werden.

\section{Problemstellung}\todo{An Methodik im Paper orientieren.}
\label{subsec:problemstellung}
Die bereits im Kapitel \ref{subsec:problematik} angesprochene Problematik soll im Zuge dieser Arbeit analysiert werden.
Dabei werden Schwachstellen identifiziert und analysiert.
Eine Analyse der Schwachstellen beinhaltet sowohl eine Einsch�tzung des Schweregrades als auch eine Aufwandssch�tzung f�r die Behebung.

Dabei werden zun�chst in Kapitel \ref{ch:basics} die notwendigen Grundlagen erl�utert.
Im nachfolgenden Kapitel \ref{ch:evaluation} wird die Evaluation der vorhandenen \ac{SNIK}-Ontologievisualisierung anhand des Papers \cite{Garcia.} durchgef�hrt.
Im abschlie�enden Kapitel \ref{ch:future} wird die Arbeit zusammengefasst und beschrieben, wie im Anschluss dieser Arbeit mit der Visualisierung von \ac{SNIK} umgegangen werden kann.
