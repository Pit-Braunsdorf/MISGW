
	\chapter{Einleitung}
	\label{ch:einleitung}
	\section{Gegenstand und Motivation}
	\label{sec:gegenstand_und_motivation}
	\subsection{Gegenstand}
	\label{subsec:gegenstand}
		In der Architektur und dem Management von Informationssystem im Gesundheitswesen werden bestimmte Begriffe regelm��ig verwendet, teilweise auch mit einer anderen Definition als �blich. Um das Verstehen der Begrifflichkeit und deren Zusammenh�nge zu vereinfachen, wurde eine Ontologie erstellt, welche diese sowohl visuell als auch schematisch darstellt. Diese Ontologie wird \ac{SNIK} genannt. 
		Der Umfang von \ac{SNIK} beruht auf verschiedenen B�chern, die jeweils eingelesen wurden. Eines dieser B�cher wird als Lehrbuch f�r die Module Architektur bzw. Management von Informationssystemen im Gesundheitswesen verwendet. Daher wird au�erdem \ac{SNIK} verwendet, um eine �bersicht zu bieten, in wie weit diese Begriffe zu einander in Beziehungen stehen.
		%TODO �blich ist doof
	\subsection{Problematik}
	\label{subsec:problematik}
		Ontologien sind Verkn�pfungen von Begriffen mit verschiedenen Beziehungen. Diese Begriffe und Beziehungen werden oftmals visuell und schematisch dargestellt. Da sehr umfangreiche Ontologien relativ schwer visuell dargestellt werden k�nnen, weil diese aufgrund der viel Zahl von Begriffen und Beziehungen sehr un�bersichtlich und komplex werden k�nnen. Diese un�bersichtliche Form kann anhand verschiedener Grunds�tze, die in dem Paper \glqq Affective Graphs: The Visual Appeal of Linked Data\grqq{} von Garcia et al. dargestellt werden, teilweise �berarbeitet werden.
	\subsection{Motivation}
	\label{subsec:motivation}
		Anhand dieser Grunds�tze kann eine Vereinfachung der Visualisierung f�r die Lehre und f�r andere Verwendungen wie beispielsweise Kommunikation mit dem Universit�tsklinikum Leipzig erfolgen. Dadurch kann die Ontologie leichter verwendet werden, um bestimmte Definitionen zu lernen bzw. um mit anderen auf Basis von diesen Definitionen zu kommunizieren. Dadurch wird es erm�glicht, dass man einen gemeinsamen Wortschatz f�r die Kommunikation verwendet. Au�erdem wird dadurch erreicht, dass verschiedene Anwendungen ebenfalls diese Ontologie verwenden k�nnen, wodurch eine einfachere Kommunikation zwischen Anwendungen und zwischen der Anwendung und einem Anwender sicher gestellt wird.
		
	\section{Problemstellung}
	\label{subsec:problemstellung}
		Die Problematik welche im Kapitel \ref{subsec:problematik} dargestellt wird, soll im Zuge dieser Arbeit auf die Ontologie \ac{SNIK} angewandt werden. Dabei soll die Visualisierung von \ac{SNIK} (einsehbar unter: \href{www.snik.eu/graph}{www.snik.eu/graph}) anhand der Grunds�tze aus dem Paper analysiert werden und entsprechende Schwachstellen identifiziert und analysiert werden. Die Analyse der Schwachstellen soll eine Einsch�tzung des Schweregrades und des gesch�tzten Behebungsaufwandes beinhalten.
