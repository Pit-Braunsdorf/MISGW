
	\chapter{Einleitung}
	\label{ch:einleitung}
	\section{Gegenstand und Motivation}
	\label{sec:gegenstand_und_motivation}
	\subsection{Gegenstand}
	\label{subsec:gegenstand}
		In der Architektur und dem Management von Informationssystem im Gesundheitswesen werden bestimmte Begriffe regelmäßig verwendet. Teilweise auch mit anderen Definitionen als in einem Lexika. Um das Verstehen der Begrifflichkeit und deren Zusammenhänge zu vereinfachen, wurde eine Ontologie erstellt, welche diese sowohl visuell als auch schematisch darstellt. Diese wird \ac{SNIK} genannt. 
		Der Umfang von \ac{SNIK} beruht auf verschiedenen Büchern, die für die Ontologie bearbeitet wurden. Eines dieser Bücher wird als Lehrbuch für die zwei Module Architektur bzw. Management von Informationssystemen im Gesundheitswesen, welche am \ac{IMISE} angeboten werden, verwendet. Daher wird \ac{SNIK} außerdem herangezogen, um eine Übersicht zu bieten, in wie weit diese Begriffe zu einander in Beziehung stehen.
		
	\subsection{Problematik}
	\label{subsec:problematik}
		Ontologien sind Verknüpfungen von Begriffen mit verschiedenen Beziehungen. Diese Begriffe und Beziehungen werden oftmals visuell und schematisch dargestellt. Sehr umfangreiche Ontologien können relativ schwer visuell dargestellt werden, weil diese aufgrund der viel Zahl von Begriffen und Beziehungen sehr unübersichtlich und komplex werden können. Diese unübersichtliche Form kann anhand verschiedener Grundsätze, die in dem Paper \glqq Affective Graphs: The Visual Appeal of Linked Data\grqq{} von Garcia et al. dargestellt werden, partiell überarbeitet werden.
	\subsection{Motivation}
	\label{subsec:motivation}
		Anhand dieser Grundsätze kann eine Vereinfachung der Visualisierung für die Lehre am \ac{IMISE} und für andere Verwendungen wie beispielsweise Kommunikation mit dem Universitätsklinikum Leipzig erfolgen. Aus diesem Grund kann die Ontologie verwendet werden, um bestimmte Definitionen zu lernen bzw. um mit anderen auf Basis von diesen Definitionen zu kommunizieren. Dadurch wird ein gemeinsamen Wortschatz für die Kommunikation ermöglicht. Außerdem wird erreicht, dass verschiedene Anwendungen auf diese Ontologie zurückgreifen können. Dadurch unkompliziertere Kommunikation zwischen Anwendungen und zwischen der Anwendung und einem Anwender sicher gestellt wird.
		
	\section{Problemstellung}
	\label{subsec:problemstellung}
		Die Problematik welche im Kapitel \ref{subsec:problematik} dargestellt wird, soll im Zuge dieser Arbeit auf die Ontologie \ac{SNIK} angewandt werden. Dabei soll die Visualisierung von \ac{SNIK} (einsehbar unter: \href{www.snik.eu/graph}{www.snik.eu/graph}) anhand der Grundsätze aus dem Paper analysiert und entsprechende Schwachstellen identifiziert und analysiert werden. Die Analyse der Schwachstellen soll eine Einschätzung des Schweregrades und des geschätzten Behebungsaufwandes beinhalten.