
\chapter{Einleitung}
\label{ch:einleitung}
\section{Gegenstand und Motivation}
\label{sec:gegenstand_und_motivation}
\subsection{Gegenstand}\todo{Der Gegenstand ist zu schwammig beschrieben. F�r die SNIK-Ontologie bitte mal die Publikationen unter \url{http://www.snik.eu/de/Publikationen/index.jsp} durchsehen und die Relevanten zitieren.}
\label{subsec:gegenstand}
In der Architektur und dem Management von Informationssystem im Gesundheitswesen werden bestimmte Begriffe regelm��ig verwendet. Teilweise auch mit anderen Definitionen als in einem Lexika.\todo{Lexika ist Plural} Um das Verstehen der Begrifflichkeit und deren Zusammenh�nge zu vereinfachen\todo{Ontologie hat noch andere Ziele und Einsatzzwecke, z.B. maschinelle Verarbeitbarkeit, Abfrage �ber SPARQL Queries}, wurde eine Ontologie erstellt, welche diese sowohl visuell als auch schematisch darstellt\todo{die Ontologie selbst stellt nichts visuell dar, das macht die Visualisierung der Ontologie}.
Diese wird \ac{SNIK} genannt.

Der Umfang von \ac{SNIK} beruht auf verschiedenen B�chern, die f�r die Ontologie bearbeitet wurden.
Eines dieser B�cher wird als Lehrbuch f�r die zwei Module Architektur bzw. Management von Informationssystemen im Gesundheitswesen, welche am \ac{IMISE} angeboten werden, verwendet.
Daher wird \ac{SNIK} au�erdem herangezogen, um eine �bersicht zu bieten, in wie weit diese Begriffe zu einander in Beziehung stehen.
Diese �bersicht ist bei \ac{SNIK} visuell dargestellt.
Die Visualisierung ist \href{http://www.snik.eu/graph}{hier} einsehbar.\todo{Die Visualisierung sollte schon etwas umfangreicher beschrieben werden, schlie�lich ist sie ja der eigentliche Gegenstand, nicht die Ontologie.}

\subsection{Problematik}\todo{Das ist noch zu schwammig. Eine umfangreiche Datenmenge f�hrt auch nicht zwangsweise zu einer un�bersichtlichen Visualisierung. Google Maps ist ja auch �bersichtlich, obwohl extrem viele Daten vorhanden sind.}
\label{subsec:problematik}
Da die Ontologie von \ac{SNIK} sehr umfangreich ist, ist die Visualisierung von \ac{SNIK} un�bersichtlich.
Durch diese Un�bersichtlichkeit stellt sich die Frage, ob diese Visualisierung von \ac{SNIK} ihr Ziel erf�llt.
Das Ziel ist eine Unterst�tzung von Studenten und Dozenten in der Lehre am \ac{IMISE} und eine vereinfachte Kommunikation\todo{Kommunikation von wem mit wem?}.
Dies kann anhand der Grunds�tze, die im Paper \glqq Affective Graphs: The Visual Appeal of Linked Data\grqq{} von Garcia et al. dargestellt werden, �berpr�ft und analysiert werden.
\subsection{Motivation}\todo{So richtig rei�t mich die Motivation noch nicht mit. Der Leser muss hier �berzeugt werden, wie wahnsinning wichtig das Thema ist, unterlegt mit belegten Zahlen.}
\label{subsec:motivation}
Anhand dieser Grunds�tze kann notfalls eine Vereinfachung der Visualisierung f�r die Lehre am \ac{IMISE} und f�r andere Verwendungen wie beispielsweise Kommunikation mit dem Universit�tsklinikum Leipzig erfolgen.
Aus diesem Grund kann die Ontologie verwendet werden, um bestimmte Definitionen zu lernen bzw. um mit anderen auf Basis von diesen Definitionen zu kommunizieren.
Dadurch wird ein gemeinsamen Wortschatz f�r die Kommunikation erm�glicht.
Au�erdem wird erreicht, dass verschiedene Anwendungen auf diese Ontologie zur�ckgreifen k�nnen.
Dadurch unkompliziertere Kommunikation zwischen Anwendungen und zwischen der Anwendung und einem Anwender sicher gestellt wird.

\section{Problemstellung}\todo{An Methodik im Paper orientieren.}
\label{subsec:problemstellung}
Die bereits im Kapitel \ref{subsec:problematik} angesprochene Problematik soll im Zuge dieser Arbeit analysiert werden.
Dabei sollen Schwachstellen identifiziert und analysiert werden.\todo{Die Einleitung wird in diesem Seminar zwar vorher geschrieben, ist am Ende aber trotzdem Teil des finalen Dokuments und kein historischer Arbeitsplan oder etwas �hnliches. Daher nicht im Stil \enquote{es soll dies und das gemacht werden} schreiben sondern auf den Rest der Arbeit verweisen, als w�rde er schon existieren. Also: \enquote{Kapitel X analysiert Schwachstellen anhand der Methodik von Y}}
Eine Analyse der Schwachstellen soll sowohl eine Einsch�tzung des Schweregrades als auch eine Aufwandssch�tzung f�r die Behebung beinhalten.

Dabei sollen zun�chst die notwendigen Grundlagen erl�utert und anschlie�end eine Evaluation der vorhandenen \ac{SNIK}-Ontologievisualisierung durchgef�hrt werden.
Abschlie�end soll noch dargestellt werden, wie weiter mit der Problematik vorgegangen werden kann.
