\chapter{Grundlagen}
\label{ch:basics}
\section{Definition Ontologie}
- Defintion suchen
\section{SNIK}
In \ac{SNIK} wird zwischen Aufgaben(functions) und Objekttypen(entity type) unterschieden. 
Die Objekttypen unterscheidet man zusätzlich noch in Passive Objekttypen(passive entity type), Rollen(roles) und Abteilungen(departments).
Weiterhin wurde das Konzept der Aufgabe wie folgt spezialisiert:\cite{Schaaf.}
\begin{itemize}
	\item Unternehmensaufgabe
	\begin{itemize}
		\item Krankenhausaufgabe
		\begin{itemize}
			\item Aufgabe der Patientenversorgung
			\item Aufgabe der Administration
			\item Aufgabe des Informationsmanagements
			\item Aufgabe der Managements
			\item Aufgabe der Ausbildung und Forschung
		\end{itemize}
	\end{itemize}
\end{itemize}

Unter dem Konzept der passiven Objekttypen versteht man \glqq{}Informationsobjekte, die von Aufgaben für deren Erfüllung benötigt oder bearbeitet werden\grqq{}. 
Die Einteilung von passiven Objekttypen orientiert sich dabei an der Spezialisierung der Aufgaben.
Dies wurde so vorgenommen, um später Aussagen zu den Krankenhausaufgaben und deren Informationsobjekte treffen zu können.\cite{Schaaf.}
\begin{itemize}
	\item Unternehmensobjekttyp
	\begin{itemize}
		\item Krankenhausobjekttyp
		\begin{itemize}
			\item Objekttyp der Patientenversorgung
			\item Objekttyp der Administration
			\item Objekttyp des Informationsmanagments
			\item Objekttyp der Managements
			\item Objekttyp der Ausbildung und Forschung
		\end{itemize}
	\end{itemize}
\end{itemize}\cite{Schaaf.}

Bei Rollen und Abteilungen wird von aktiven Objekttypen gesprochen, da diese aktiv in Beziehung zu den passiven Objekttypen bzw. Aufgaben stehen.\cite{Schaaf.}

Zu jedem Objekttyp können Annotationen hinterlegt werden. 
Annontationen werden verwendet um Informationen zu den Objekttypen zu hinterlegen.\cite{Schaaf.}

Zusätzlich zu den Konzepten sind auch Relationen definiert wurden, welche bei Bedarf erweitert werden können.\cite{Schaaf.}
\section{Affective Graphs: The Visual Appeal of Linked Data}
Nach Garcia et al ist das Ziel des eine Alternative aufzeigen wie man gut designte und ästhetische Linked-Data Produkte erstellen kann, ohne das ein Informationsverlust damit einhergeht.
Dabei wird zu nächst das bisherige vorgestellt und in wie weit die Prinzipien, die im Laufe des Papers aufgestellt werden, auf Semantic Web-Produkte anwendbar sind.\cite[S. 2]{Garcia.}
Diese Prinzipien für die Datenvisualisierung im Semantic Web:
\begin{enumerate}
	\item Hierarchie zwischen den Elementen müssen klar erkennbar sein.
	\item Kindknoten sollten in unmittelbarer Nähe zu deren Elternknoten positioniert sein.
	\item Semantisch zusammenhängende Knoten sollten gemeinsam positioniert werden.
	\item Überschneidung und Überlappung von Kanten sollten vermieden werden.
	\item Kommentare sind in der Nähe der dazugehörigen Knoten platziert werden.
	\item Verzierungen, wie Labels und Icons, sind in der Nähe der dazugehörigen Knoten positioniert werden.
	\item Einschränkungen beim Zeichnen des Graphen (bspw. Symmetrie, kompaktes Zeichnen) sollten beachtet werden.
\end{enumerate}

Anhand dieser Prinzipien für die Datenvisualisierung wurden die 19 abschließenden Prinzipien des Papers aufgestellt. 

%TODO Bilder einfügen.

Im Paper wird außerdem eine Methodik vorgestellt wie Probleme identifiziert und kategorisiert werden können. 
Dabei wird jeder Punkt auf einer Skala von 0 bis 1 bewertet, dabei bedeutet 0 sehr schlecht und 1 sehr gut.
Durch diese Skala von sehr schlecht bis sehr gut können Punkte aufgegriffen werden, welche bereits gut umgesetzt aber verbesserungswürdig sind. 
Außerdem kann man anhand dieser Skala und einer regelmäßigen Neubewertung einen Verlauf erkennen und entsprechend reagieren.

Mit der Darstellung einer möglichen Vorgehensweise anhand eines Beispiels ist dieses Paper sehr gut geeignet, um die Visualisierung von \ac{SNIK} zu verbessern. 


-Vorstellung einer Methodik um Probleme zu identifizieren
-Darstellung anhand eines Beispiels
-0 bis 1 
