\chapter{Evaluation der Visualisierung von SNIK}
\label{ch:evaluation}

%Codierung anhand des Papers [3]
%0.0 => schlecht 
%1.0 => gut

%Gut: 
%-unterschiedliche Farben f�r unterschiedliche Objekte | 0.9 %TODO Welche Farbe bedeutet was?
%-unterschiedliche Gr��en bei Objekten | 0.8 %TODO vielleicht genauer erkl�ren
%-Drag and Drop von Knoten => einfache M�glichkeit sich die �bersicht selbst anzuordnen | 0.9
%-SPARQL an sich vorhanden | 0.7

%Negativ: 
%-keine farbliche Kennzeichnung f�r Beziehung                | 0
%-keine erkennbarkeit in welche Richtung die Beziehung gehen | 0
%-keine andere Darstellung von hierarchischen Beziehungen    | 0.2
%-keine M�glichkeit die Abfrage zu verfeinern mittels SparQL | 0.4
%-kein sinnvolles Zooming                                    | 0.4

%sonstiges:
%	-fehler beim Hovering vom Menue
%	-Menue Punkte klappen sich nicht wieder zu

\section{Analyse des Aufbaus der Website}
%TODO einf�gen der aktuellen Darstellung
Der aktuelle Aufbau ist relativ un�bersichtlich und nicht sehr komfortabel, beispielsweise ist es nicht ersichtlich wie man die Definition der Objekte aufrufen kann. (Vermutung: Rechtsklick => Richtung Description Alternative: Doppelklick) Au�erdem �ffnet sich die Definition der Beschreibung in einem PopUp-Fenster, was viele Standardtechnisch unterdr�cken. Als Alternative k�nnte man einen modalen Dialog vorsehen.	

%TODO Einf�gen einer Planskizze
	
\section{Analyse der Beziehungen zwischen Objekten}
Die Darstellung der Beziehungen zwischen den Objekten ist vollst�ndig gleich. Das hei�t es existiert keine unterschiedliche Darstellung trotz unterschiedlicher Arten von Beziehungen. Beispielsweise k�nnte man die Darstellung von hierarchischen Beziehungen wie \glqq is A\grqq{} oder \glqq subclass of\grqq{} durch eine dreiecks-Linie darstellen.
%TODO andere Bezeichnung f�r dreiecks-Linie 

Andere Beziehungen zwischen zwei Objekten sollten farblich codiert werden, so dass man anhand der Farbe direkt auf den Typen der Beziehung zur�ckschlie�en kann. 