\chapter{Evaluation der Visualisierung von SNIK}
\label{ch:evaluation}

%Codierung anhand des Papers [3]
%0.0 => schlecht 
%1.0 => gut

%Gut: 
%-unterschiedliche Farben f�r unterschiedliche Objekte | 0.9 %TODO Welche Farbe bedeutet was?
%-unterschiedliche Gr��en bei Objekten | 0.8 %TODO vielleicht genauer erkl�ren
%-Drag and Drop von Knoten => einfache M�glichkeit sich die �bersicht selbst anzuordnen | 0.9
%-SPARQL an sich vorhanden | 0.7

%Negativ: 
%-keine farbliche Kennzeichnung f�r Beziehung                | 0
%-keine erkennbarkeit in welche Richtung die Beziehung gehen | 0
%-keine andere Darstellung von hierarchischen Beziehungen    | 0.2
%-keine M�glichkeit die Abfrage zu verfeinern mittels SparQL | 0.4
%-kein sinnvolles Zooming                                    | 0.4

%sonstiges:
%	-fehler beim Hovering vom Menue
%	-Menue Punkte klappen sich nicht wieder zu
\section{Allgemeine Analyse der Visualisierung}
Zun�chst soll die Visualisierung allgemein analysiert werden. 
Dabei wird auf die Methodik, die im Paper \cite{Garcia.} vorgestellt wurde, zur�ckgegriffen. 
Die Analyse ist in der Tabelle ..., dargestellt.

\begin{tabular}[h]{|l|l|l|}
	\hline
	Beschreibung & Bewertung & Begr�ndung \\
	Unterschiedliche Knotentypen werden in unterschiedlichen Farben dargestellt & 0.9 & Die Farbkodierung �bermittelt einfacher Informationen des Types (Aufgabe, Objekttyp). Dies entspricht den Prinzip 1 sowie Prinzip 6.)\\
	Mittels Drag and Drop von Knoten kann man sich eine eigene �bersicht erstellen. & 0.9 & Dadurch k�nnen Begriffe, die f�r den aktuellen Sachverhalt ben�tigt werden individuell zusammengef�hrt werden. Dies entspricht dem Prinzip 9.\\
	Durch einen Link in der Visualisierung kann man �ber SPARQL Abfragen generieren, man kann keine bereits erstellte Abfragen aus der Visualisierung genauer spezifizieren. & 0.4 & Die Spezifizierung von Abfragen wird vor allem von Experten ben�tigt, weswegen das Fehlen nicht so schwerwiegend ist. Dennoch sollte dies entsprechend Prinzip 9 umgesetzt werden. \\
	Fehlende Kennzeichnung f�r Beziehungen, um entsprechende Beziehungen schnell erkennen zu k�nnen. & 0 & Analog zu der Farbkodierung bei Knotentypen sollten auch Beziehungen kodiert werden. Beispielsweise k�nnte die Kodierung durch unterschiedliche Darstellungen, Farben oder einem Label erm�glicht werden. Dies w�rde ebenfalls dem Prinzip 1 entsprechen. \\
	Keine spezielle Darstellung von hierarchischen Beziehungen. & 0.2 &  \\
\end{tabular}

\section{Analyse des Aufbaus der Website}
%TODO einf�gen der aktuellen Darstellung
Der aktuelle Aufbau ist relativ un�bersichtlich und nicht sehr komfortabel, beispielsweise ist es nicht ersichtlich wie man die Definition der Objekte aufrufen kann. Au�erdem �ffnet sich die Definition der Beschreibung in einem PopUp-Fenster, was viele Standardtechnisch unterdr�cken. Als Alternative k�nnte man einen modalen Dialog vorsehen.	

%TODO Einf�gen einer Planskizze
	
\section{Analyse der Beziehungen zwischen Objekten}
Die Darstellung der Beziehungen zwischen den Objekten ist vollst�ndig gleich. Das hei�t es existiert keine unterschiedliche Darstellung trotz unterschiedlicher Arten von Beziehungen. Beispielsweise k�nnte man die Darstellung von hierarchischen Beziehungen wie \glqq is A\grqq{} oder \glqq subclass of\grqq{} durch eine dreiecks-Linie darstellen.
%TODO andere Bezeichnung f�r dreiecks-Linie 

Andere Beziehungen zwischen zwei Objekten sollten farblich codiert werden, so dass man anhand der Farbe direkt auf den Typen der Beziehung zur�ckschlie�en kann. 